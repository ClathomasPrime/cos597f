%%macros for Advanced Algorthm Design; Fall 2008, Sanjeev Arora
\newcommand\p{\mbox{\bf P}\xspace}
\newcommand\np{\mbox{\bf NP}\xspace}

%%Computational problems

\newcommand\sat{\mbox{SAT}\xspace}

\newcommand\numsat{\mbox{$\sharp$ SAT}\xspace}
%% Notation for integers, natural numbers, reals, fractions, sets, cardinalities
%%and so on

\newcommand\bz{\mbox{\bf Z}}
\newcommand\nat{\mbox{\bf N}}
\newcommand\rea{\mbox{\bf R}}
\newcommand\B{\{0,1\}}      % boolean alphabet  use in math mode
\newcommand\Bs{\{0,1\}^*}   % B star            use in math mode
\newcommand\true{\mbox{\sc True}}
\newcommand\false{\mbox{\sc False}}
\DeclareRobustCommand{\fracp}[2]{{#1 \overwithdelims()#2}}
\DeclareRobustCommand{\fracb}[2]{{#1 \overwithdelims[]#2}}
\newcommand{\marginlabel}[1]%
{\mbox{}\marginpar{\it{\raggedleft\hspace{0pt}#1}}}
\newcommand\card[1]{\left| #1 \right|} %cardinality of set S; usage \card{S}
\newcommand\set[1]{\left\{#1\right\}} %usage \set{1,2,3,,}
\newcommand\poly{\mbox{poly}}  %usage \poly(n)
\newcommand{\floor}[1]{\lfloor\, $#1$\,\rfloor}
\newcommand{\ceil}[1]{\lceil\, $#1$\,\rceil}
\newcommand{\comp}[1]{\overline{#1}}
\newcommand{\defeq}{\overset{\mathrm{def}}{=}}
\newcommand{\bits}{\set{0,1}}
%% Various things to write in small caps

\newcommand{\disclaimer}{\textbf{These notes may contain errors, and are not intended for use as a primary source.} I have included references at the end of the notes for those who wish detail surrounding skipped steps.}

\def\scand{\mbox{\sc and}}
\def\scor{\mbox{\sc or}}
\def\scnot{\mbox{\sc not}}
\def\scyes{\mbox{\sc yes}}


%% short-hands for relational simbols

%\newcommand\to{\rightarrow} 
\newcommand{\from}{:}
\newcommand\xor{\oplus}
\newcommand\bigxor{\bigoplus}
\newcommand{\logred}{\leq_{\log}}
\def\iff{\Leftrightarrow}
\def\implies{\Rightarrow}


%% probability stuff

\newcommand\pr{\mathop{\mbox{\bf Pr}}}
\newcommand\av{\mathop{\mbox{\bf E}}}
\newcommand\var{\mathop{\mbox{\bf Var}}}


%% macros to write pseudo-code

\newlength{\pgmtab}  %  \pgmtab is the width of each tab in the
\setlength{\pgmtab}{1em}  %  program environment
 \newenvironment{program}{\renewcommand{\baselinestretch}{1}%
\begin{tabbing}\hspace{0em}\=\hspace{0em}\=%
\hspace{\pgmtab}\=\hspace{\pgmtab}\=\hspace{\pgmtab}\=\hspace{\pgmtab}\=%
\hspace{\pgmtab}\=\hspace{\pgmtab}\=\hspace{\pgmtab}\=\hspace{\pgmtab}\=%
\+\+\kill}{\end{tabbing}\renewcommand{\baselinestretch}{\intl}}
\newcommand {\BEGIN}{{\bf begin\ }}
\newcommand {\ELSE}{{\bf else\ }}
\newcommand {\IF}{{\bf if\ }}
\newcommand {\FOR}{{\bf for\ }}
\newcommand {\TO}{{\bf to\ }}
\newcommand {\DO}{{\bf do\ }}
\newcommand {\WHILE}{{\bf while\ }}
\newcommand {\ACCEPT}{{\bf accept}}
\newcommand {\REJECT}{\mbox{\bf reject}}
\newcommand {\THEN}{\mbox{\bf then\ }}
\newcommand {\END}{{\bf end}}
\newcommand {\RETURN}{\mbox{\bf return\ }}
\newcommand {\HALT}{\mbox{\bf halt}}
\newcommand {\REPEAT}{\mbox{\bf repeat\ }}
\newcommand {\UNTIL}{\mbox{\bf until\ }}
\newcommand {\TRUE}{\mbox{\bf true\ }}
\newcommand {\FALSE}{\mbox{\bf false\ }}
\newcommand {\FORALL}{\mbox{\bf for all\ }}
\newcommand {\DOWNTO}{\mbox{\bf down to\ }}

%formatting commands
\newcounter{lecnum}
\newcommand{\lecture}[3]{%
\pagestyle{myheadings}%
\thispagestyle{plain}%
\newpage %
\setcounter{lecnum}{#1}%
\setcounter{page}{1}%
%\rule{\linewidth}{1mm}
\begin{center} 
\shadowbox{\parbox{4.3in}{{\small {\bf {\sc princeton univ. F'18
\hfill cos 597F: Open Problems in AGT}}}

\medskip

\begin{center}
{\large Lecture #1: #2}
\end{center}

\medskip

Lecturer: {\em Matt Weinberg} \hfill Scribe:{\em #3}
\vspace{2pt}}}
\vspace{6pt}\end{center}}


%%%%%%%%%%%%%%%%%%%%%%%%%%%%%%%%%%%%%%%%%%%%%%%%%%%%%%%%%%%%%%%%%%%%%%%%%%%
%%%%%%%%%%%%%%%%%%%%%%%%%%%%%%%%%%%%%%%%%%%%%%%%%%%%%%%%%%%%%%%%%%%%%%%%%%%




\newlength{\tpush}
\setlength{\tpush}{2\headheight}
\addtolength{\tpush}{\headsep}

%       Usage: \htitle{title}{datelec}{dateout}
\newcommand{\htitle}[4]{\noindent\vspace*{-\tpush}\newline\parbox{\textwidth}
{U.C. Berkeley --- \coursenum : \coursename \hfill #1 \newline
\courseprof \hfill #2 \newline
\mbox{#4}\hfill Last revised #3 \newline
\mbox{}\hrulefill\mbox{}}\vspace*{1ex}\mbox{}\newline
\bigskip
\begin{center}{\Large\bf #1}\end{center}
\bigskip}

%       Usage: \handout{title}{datelec}{dateout}{scribe}
\newcommand{\handout}[4]{\thispagestyle{empty}
 \markboth{Notes for #1: #2}{Notes for #1: #2}
 \pagestyle{myheadings}\htitle{#1}{#2}{#3}{#4}}

\newcommand{\htitlewithouttitle}[2]{\noindent\vspace*{-\tpush}\newline\parbox{6.5in}{\coursenumber : \coursetitle
\hfill Columbia University\newline
Handout #1\hfill#2\vspace*{-.5ex}\newline
\mbox{}\hrulefill\mbox{}}\vspace*{1ex}\mbox{}\newline}

\newcommand{\handoutwithouttitle}[2]{\thispagestyle{empty}
 \markboth{Handout #1}{Handout #1}
 \pagestyle{myheadings}\htitlewithouttitle{#1}{#2}}


%%%%%%%%%%%%%%%%%%%%%%%%%%%%%%%%%%%%%%%%%%%%%%%%%%%%%%%%%%%%%%%%%
%%% Commands to include figures


%% PSfigure
\iffalse
\newcommand{\PSfigure}[3]{\begin{figure}[t] 
  \centerline{\vbox to #2 {\vfil \psfig{figure=#1.eps,height=#2} }} 
  \caption{#3}\label{#1} 
  \end{figure}} 
\newcommand{\twoPSfigures}[5]{\begin{figure*}[t]
  \centerline{%
    \hfil
    \begin{tabular}{c}
        \vbox to #3 {\vfil\psfig{figure=#1.eps,height=#3}} \\ (a)
    \end{tabular}
    \hfil\hfil\hfil
    \begin{tabular}{c}
        \vbox to #3 {\vfil\psfig{figure=#2.eps,height=#3}} \\ (b)
    \end{tabular}
    \hfil}
  \caption{#4}
  \label{#5}
%  \sublabel{#1}{(a)}
%  \sublabel{#2}{(b)}
  \end{figure*}}

\fi
\newcounter{fignum}

% fig
%command to insert figure. usage \fig{name}{h}{caption}
%where name.eps is the postscript file and h is the height in inches
%The figure is can be referred to using \ref{name}
\newcommand{\fig}[3]{%
\begin{minipage}{\textwidth}
\centering\epsfig{file=#1.eps,height=#2}
\caption{#3} \label{#1}
\end{minipage}
}%

\newcommand{\myfig}[4]{%
%\begin{minipage}{\textwidth}
\begin{figure}[h]
\begin{center} 
\includegraphics[height=#2]{#1}
\newline
\caption{#3} \label{#4}
\end{center} 
\end{figure}
%\end{figure}
%\end{minipage}
}%
% ffigure
% Usage: \ffigure{name of file}{height}{caption}{label}
\newcommand{\ffigure}[4]{\begin{figure} 
  \centerline{\vbox to #2 {\hfil \psfig{figure=#1.eps,height=#2} }} 
  \caption{#3}\label{#4} 
  \end{figure}} 

% ffigureh
% Usage: \ffigureh{name of file}{height}{caption}{label}
\newcommand{\ffigureh}[4]{\begin{figure}[!h] 
  \centerline{\vbox to #2 {\vfil \psfig{figure=#1.eps,height=#2} }} 
  \caption{#3}\label{#4} 
  \end{figure}} 



