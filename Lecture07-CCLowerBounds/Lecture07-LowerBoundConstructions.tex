%%version Feb 3, 2001 by SA
%% modified Feb 12, 2001 by LT (smaller margins, 11pt font)

\documentclass[11pt]{article}
%\usepackage{lucidbry}
\usepackage{amsmath}
%\usepackage{trees}
\usepackage{ntheorem}
\usepackage{latexsym}
\usepackage{xspace}
\usepackage{amssymb}
\usepackage{fancybox}
\usepackage{graphicx}
%\usepackage{psfig}
%\usepackage{lucmath}
%tTHEOREM TYPE ENVIRONMENTS
\theoremstyle{break} 
\theoremheaderfont{\scshape}
\theorembodyfont{\slshape}
\newtheorem{Thm}{Theorem}
\newtheorem{Lem}[Thm]{Lemma}
\newtheorem{Cor}[Thm]{Corollary}
\newtheorem{Prop}[Thm]{Proposition}
\newtheorem{Obs}{Observation}
\newtheorem{theorem}{Theorem}
\newtheorem{lemma}[Thm]{Lemma}
\newtheorem{corollary}[Thm]{Corollary}
\newtheorem{proposition}[Thm]{Proposition}
\newtheorem{observation}{Observation}
\theoremstyle{plain} 
\theorembodyfont{\rmfamily} 
\newtheorem{Ex}{Exercise}
\newtheorem{Exa}{Example}
\newtheorem{Rem}{Remark}
\theorembodyfont{\itshape}
\newtheorem{Def}{Definition}
\newtheorem{definition}{Definition}
\newenvironment{proof}{\noindent {\sc Proof:}}{$\Box$ \medskip} 
\newenvironment{problems} % Definition of problems
 {\renewcommand{\labelenumi}{\S\theenumi}
	\begin{enumerate}}{\end{enumerate}}

\setlength{\oddsidemargin}{0pt}
\setlength{\evensidemargin}{0pt}
\setlength{\textwidth}{6.0in}
\setlength{\topmargin}{0in}
\setlength{\textheight}{8.5in}






%%macros for Advanced Algorthm Design; Fall 2008, Sanjeev Arora
\newcommand\p{\mbox{\bf P}\xspace}
\newcommand\np{\mbox{\bf NP}\xspace}

%%Computational problems

\newcommand\sat{\mbox{SAT}\xspace}

\newcommand\numsat{\mbox{$\sharp$ SAT}\xspace}
%% Notation for integers, natural numbers, reals, fractions, sets, cardinalities
%%and so on

\newcommand\bz{\mbox{\bf Z}}
\newcommand\nat{\mbox{\bf N}}
\newcommand\rea{\mbox{\bf R}}
\newcommand\B{\{0,1\}}      % boolean alphabet  use in math mode
\newcommand\Bs{\{0,1\}^*}   % B star            use in math mode
\newcommand\true{\mbox{\sc True}}
\newcommand\false{\mbox{\sc False}}
\DeclareRobustCommand{\fracp}[2]{{#1 \overwithdelims()#2}}
\DeclareRobustCommand{\fracb}[2]{{#1 \overwithdelims[]#2}}
\newcommand{\marginlabel}[1]%
{\mbox{}\marginpar{\it{\raggedleft\hspace{0pt}#1}}}
\newcommand\card[1]{\left| #1 \right|} %cardinality of set S; usage \card{S}
\newcommand\set[1]{\left\{#1\right\}} %usage \set{1,2,3,,}
\newcommand\poly{\mbox{poly}}  %usage \poly(n)
\newcommand{\floor}[1]{\lfloor\, $#1$\,\rfloor}
\newcommand{\ceil}[1]{\lceil\, $#1$\,\rceil}
\newcommand{\comp}[1]{\overline{#1}}
\newcommand{\defeq}{\overset{\mathrm{def}}{=}}
\newcommand{\bits}{\set{0,1}}
%% Various things to write in small caps

\newcommand{\disclaimer}{\textbf{These notes may contain errors, and are not intended for use as a primary source.} I have included references at the end of the notes for those who wish detail surrounding skipped steps.}

\def\scand{\mbox{\sc and}}
\def\scor{\mbox{\sc or}}
\def\scnot{\mbox{\sc not}}
\def\scyes{\mbox{\sc yes}}


%% short-hands for relational simbols

%\newcommand\to{\rightarrow} 
\newcommand{\from}{:}
\newcommand\xor{\oplus}
\newcommand\bigxor{\bigoplus}
\newcommand{\logred}{\leq_{\log}}
\def\iff{\Leftrightarrow}
\def\implies{\Rightarrow}


%% probability stuff

\newcommand\pr{\mathop{\mbox{\bf Pr}}}
\newcommand\av{\mathop{\mbox{\bf E}}}
\newcommand\var{\mathop{\mbox{\bf Var}}}


%% macros to write pseudo-code

\newlength{\pgmtab}  %  \pgmtab is the width of each tab in the
\setlength{\pgmtab}{1em}  %  program environment
 \newenvironment{program}{\renewcommand{\baselinestretch}{1}%
\begin{tabbing}\hspace{0em}\=\hspace{0em}\=%
\hspace{\pgmtab}\=\hspace{\pgmtab}\=\hspace{\pgmtab}\=\hspace{\pgmtab}\=%
\hspace{\pgmtab}\=\hspace{\pgmtab}\=\hspace{\pgmtab}\=\hspace{\pgmtab}\=%
\+\+\kill}{\end{tabbing}\renewcommand{\baselinestretch}{\intl}}
\newcommand {\BEGIN}{{\bf begin\ }}
\newcommand {\ELSE}{{\bf else\ }}
\newcommand {\IF}{{\bf if\ }}
\newcommand {\FOR}{{\bf for\ }}
\newcommand {\TO}{{\bf to\ }}
\newcommand {\DO}{{\bf do\ }}
\newcommand {\WHILE}{{\bf while\ }}
\newcommand {\ACCEPT}{{\bf accept}}
\newcommand {\REJECT}{\mbox{\bf reject}}
\newcommand {\THEN}{\mbox{\bf then\ }}
\newcommand {\END}{{\bf end}}
\newcommand {\RETURN}{\mbox{\bf return\ }}
\newcommand {\HALT}{\mbox{\bf halt}}
\newcommand {\REPEAT}{\mbox{\bf repeat\ }}
\newcommand {\UNTIL}{\mbox{\bf until\ }}
\newcommand {\TRUE}{\mbox{\bf true\ }}
\newcommand {\FALSE}{\mbox{\bf false\ }}
\newcommand {\FORALL}{\mbox{\bf for all\ }}
\newcommand {\DOWNTO}{\mbox{\bf down to\ }}

%formatting commands
\newcounter{lecnum}
\newcommand{\lecture}[3]{%
\pagestyle{myheadings}%
\thispagestyle{plain}%
\newpage %
\setcounter{lecnum}{#1}%
\setcounter{page}{1}%
%\rule{\linewidth}{1mm}
\begin{center} 
\shadowbox{\parbox{4.3in}{{\small {\bf {\sc princeton univ. F'18
\hfill cos 597F: Open Problems in AGT}}}

\medskip

\begin{center}
{\large Lecture #1: #2}
\end{center}

\medskip

Lecturer: {\em Matt Weinberg} \hfill Scribe:{\em #3}
\vspace{2pt}}}
\vspace{6pt}\end{center}}


%%%%%%%%%%%%%%%%%%%%%%%%%%%%%%%%%%%%%%%%%%%%%%%%%%%%%%%%%%%%%%%%%%%%%%%%%%%
%%%%%%%%%%%%%%%%%%%%%%%%%%%%%%%%%%%%%%%%%%%%%%%%%%%%%%%%%%%%%%%%%%%%%%%%%%%




\newlength{\tpush}
\setlength{\tpush}{2\headheight}
\addtolength{\tpush}{\headsep}

%       Usage: \htitle{title}{datelec}{dateout}
\newcommand{\htitle}[4]{\noindent\vspace*{-\tpush}\newline\parbox{\textwidth}
{U.C. Berkeley --- \coursenum : \coursename \hfill #1 \newline
\courseprof \hfill #2 \newline
\mbox{#4}\hfill Last revised #3 \newline
\mbox{}\hrulefill\mbox{}}\vspace*{1ex}\mbox{}\newline
\bigskip
\begin{center}{\Large\bf #1}\end{center}
\bigskip}

%       Usage: \handout{title}{datelec}{dateout}{scribe}
\newcommand{\handout}[4]{\thispagestyle{empty}
 \markboth{Notes for #1: #2}{Notes for #1: #2}
 \pagestyle{myheadings}\htitle{#1}{#2}{#3}{#4}}

\newcommand{\htitlewithouttitle}[2]{\noindent\vspace*{-\tpush}\newline\parbox{6.5in}{\coursenumber : \coursetitle
\hfill Columbia University\newline
Handout #1\hfill#2\vspace*{-.5ex}\newline
\mbox{}\hrulefill\mbox{}}\vspace*{1ex}\mbox{}\newline}

\newcommand{\handoutwithouttitle}[2]{\thispagestyle{empty}
 \markboth{Handout #1}{Handout #1}
 \pagestyle{myheadings}\htitlewithouttitle{#1}{#2}}


%%%%%%%%%%%%%%%%%%%%%%%%%%%%%%%%%%%%%%%%%%%%%%%%%%%%%%%%%%%%%%%%%
%%% Commands to include figures


%% PSfigure
\iffalse
\newcommand{\PSfigure}[3]{\begin{figure}[t] 
  \centerline{\vbox to #2 {\vfil \psfig{figure=#1.eps,height=#2} }} 
  \caption{#3}\label{#1} 
  \end{figure}} 
\newcommand{\twoPSfigures}[5]{\begin{figure*}[t]
  \centerline{%
    \hfil
    \begin{tabular}{c}
        \vbox to #3 {\vfil\psfig{figure=#1.eps,height=#3}} \\ (a)
    \end{tabular}
    \hfil\hfil\hfil
    \begin{tabular}{c}
        \vbox to #3 {\vfil\psfig{figure=#2.eps,height=#3}} \\ (b)
    \end{tabular}
    \hfil}
  \caption{#4}
  \label{#5}
%  \sublabel{#1}{(a)}
%  \sublabel{#2}{(b)}
  \end{figure*}}

\fi
\newcounter{fignum}

% fig
%command to insert figure. usage \fig{name}{h}{caption}
%where name.eps is the postscript file and h is the height in inches
%The figure is can be referred to using \ref{name}
\newcommand{\fig}[3]{%
\begin{minipage}{\textwidth}
\centering\epsfig{file=#1.eps,height=#2}
\caption{#3} \label{#1}
\end{minipage}
}%

\newcommand{\myfig}[4]{%
%\begin{minipage}{\textwidth}
\begin{figure}[h]
\begin{center} 
\includegraphics[height=#2]{#1}
\newline
\caption{#3} \label{#4}
\end{center} 
\end{figure}
%\end{figure}
%\end{minipage}
}%
% ffigure
% Usage: \ffigure{name of file}{height}{caption}{label}
\newcommand{\ffigure}[4]{\begin{figure} 
  \centerline{\vbox to #2 {\hfil \psfig{figure=#1.eps,height=#2} }} 
  \caption{#3}\label{#4} 
  \end{figure}} 

% ffigureh
% Usage: \ffigureh{name of file}{height}{caption}{label}
\newcommand{\ffigureh}[4]{\begin{figure}[!h] 
  \centerline{\vbox to #2 {\vfil \psfig{figure=#1.eps,height=#2} }} 
  \caption{#3}\label{#4} 
  \end{figure}} 





\theoremstyle{plain}
\theorembodyfont{\rmfamily} 
\newtheorem*{claim}{Claim:}

\begin{document}

\lecture{7}{Communication Lower Bounds}{Clay Thomas}
\disclaimer 

\section{Logistics}
  First homework will be posted soon. 

\section{Outline}
  This class will just present a list of constructions proving communication lower
  bounds in various settings. Many of them are simple, few of them are advanced
  and may skip details. Just in case some people haven't seen communication
  complexity before, I'll quickly go through the basic arguments to start.

  \begin{definition} A two-party communication problem consists of a boolean
  function $f:\{0,1\}^a \times \{0,1\}^b \rightarrow \{0,1\}$ (for this entire
  lecture, let $a = b = k$). Alice is given some input $A \in \{0,1\}^a$ (but
  doesn't know anything about $B$), and Bob is given some input $B \in \{0,1\}^b$
  (but doesn't know anything about $A$). Their goal is to compute $f(A, B)$.
  \end{definition} \begin{definition} 
    A \emph{deterministic communication
    protocol} for $f$ specifies for Alice, as a function of her input $A$ and all
    messages sent so far $a_1, b_1,\ldots, a_\ell, b_\ell$, what message
    $a_{\ell+1}$ to send next. It also specifies for Bob, as a function of his
    input $B$ and all messages sent so far what message $b_{\ell+1}$ to send next.
    We'll be interested in the \emph{communication complexity} of a protocol: the
    maximum number of bits ever sent in total between Alice and Bob.
  \end{definition}

  You can take the final output of a protocol as 
  a function of the entire transaction
  history $a_1, b_1, \ldots, a_\ell, b_\ell$.
  Alternatively, you can stop communication once either party knows the
  answer (or communicate that answer back with one extra bit).
  The exact details on this point don't matter.

  \begin{Exa}[Equality] 
    Let $f(A, B) = 1$ if and only if $A = B$. One protocol to
    solve this is the following: Alice and Bob both output their entire string in
    the first round, and declare $1$ if and only if their outputs match. A slightly
    more clever protocol might have Alice and Bob each output the first bit of their
    string and see if they match. If so, continue. If not, output $0$. If they make
    it all the way to the end without stopping, output $1$.
  \end{Exa}

  \begin{Exa}[Disjointness] For this problem, it will make more sense to think of
  $A$ as a subset of $[k]$ (the coordinates which are $1$), and $B$ as a subset of
  $[k]$. Then consider $f(A, B) = 1$ if and only if $A \cap B = \emptyset$. One
  protocol to solve this is the following: Alice and Bob both output their entire
  set, and declare $1$ if and only if their outputs match.\end{Exa}

  Both examples are core problems for communication complexity. Interestingly, the
  natural procedures we posed seem to basically communicate the entire input in
  the worst case. It turns out that this is because the protocols are (almost)
  optimal (among deterministic protocols). We'll prove this using a generic type
  of argument, called a \emph{fooling set argument}. The main idea is the
  following: Imagine for instance that the complete transcript (that is, all
  messages sent by Alice and all messages sent by Bob, in order) are identical for
  inputs $(A_1, B_1)$ and $(A_2, B_2)$. Then we claim that the transcripts must
  also be identical for $(A_1, B_2)$ and $(A_2, B_1)$.

  \begin{observation}[Rectangles] 
    Let $T(A, B)$ denote the complete transcript
    (all messages sent, in order, and their final answers) produced by Alice and Bob
    using a communication protocol on input $(A, B)$.
    Suppose $T(A_1, B_1) = T(A_2, B_2) = T$.
    Then $T(A_2, B_1) = T = T(A_1, B_2)$ as well,
    and thus the protocol gives the same answer on all four input pairs.
  \end{observation} 
  \begin{proof} 
    Assume for contradiction that this were not the
    case, and w.l.o.g. say that the different transcript is $T(A_1, B_2) \neq
    T(A_2, B_2)$. Because the protocol is deterministic, and Bob has the same
    input in both instances, Bob will continue outputting the same message every
    round until Alice sends a different message. Therefore, in order for
    $T(A_1, B_2)$ to possibly not equal $T(A_2, B_2)$,
    it must be the case that Alice sends the first different message.
    Let $T$ be the transcript prior to the first different message.
    Because $T(A_2, B_2) = T(A_1, B_1)$, and Alice's next message can
    depend only on $T$ and her own input,
    Alice must send the same next message if her input is $A_1$ or $A_2$.
    This is a contradiction, so the transcripts must indeed be equal.
  \end{proof}

  The above is called a ``rectangle argument'' because if we were to draw a matrix
  where the rows are Alice's input and columns are Bob's, it means that entries
  that share the same transcript must form a rectangle (okay, not exactly, but for
  every transcript $T$, there exists a relabeling of the rows/columns so that the
  pairs that induce this transcript are a rectangle). It seems quite simple but is
  surprisingly powerful because if protocols use limited communication, it must be
  the case that multiple inputs have the same transcript
  (by the pigeonhole principle), and the rectangle
  argument lets us conclude that other inputs necessarily use the same transcript
  as well. 

  \begin{definition} A \emph{fooling set} for the function $f$ is a set $F
  \subseteq \{0,1\}^k \times \{0,1\}^k$ such that: \begin{itemize} \item $f(A, B)
      = 1$ for all $(A, B) \in F$.  \item For every $(A_1, B_1), (A_2, B_2) \in
        F$, either $f(A_1, B_2) \neq 1$ or $f(A_2, B_1) \neq 1$ (or both).
    \end{itemize} \end{definition}

  \begin{proposition}
    Let $F$ be a fooling set for $f$. Then the deterministic
    communication complexity of $f$ is at least $\log_2(|F|)$.  
  \end{proposition}
  \begin{proof} 
    Assume for contradiction that there exists a deterministic
    protocol with communication $< \log_2|F|$.
    There are less than $2^{\log_2|F|}=|F|$ possible transactions such a
    protocol can use.
    Thus, by the pigeonhole principle,
    there exists some $(A_1, B_1), (A_2, B_2) \in F$ such that 
    $T(A_1, B_1) = T(A_2, B_2)$.
    By the rectangle argument, this necessarily implies $T(A_1, B_2)
    = T(A_2, B_1) = T(A_1, B_1)$ as well. 
    Because $f$ is completely determined by the transaction history,
    we must have $f(A_1, B_2) = f(A_2, B_1) = 1$ as well.

    However, this directly contradicts the second property of fooling sets.
  \end{proof}

  So if a function admits a large fooling set, it is provably impossible to have a
  short deterministic communication protocol. The remaining work is just to show
  that both Equality and Disjointness admit large fooling sets.

  \begin{proposition} Equality admits a fooling set of size $2^k$. 

  \end{proposition} \begin{proof} 
    Let $F = \{(X, X)\ |\ X \in \{0,1\}^k \}$.
    Then $f(A, B) = 1$ for all $(A, B) \in F$. But $F(A_1, B_2) = 0$ for
    all $A_1 \neq B_2$. So the second property of the fooling set is satisfied.
  \end{proof}

  Notice that after unraveling all the definitions, the proof is essentially
  saying the following: if you had a deterministic protocol that used $< k$ bits,
  there are two inputs of the form $(X, X)$, $(Y, Y)$ that are treated exactly the
  same. This protocol must therefore also treat $(X, Y)$ and $(Y, X)$ the same as
  $(X, X)$, even though the difference affects the value of $f$.
  That is, the protocol is ``fooled''
  into thinking the input is $(X, X)$ or $(Y, Y)$ when it really could be $(X, Y)$
  or $(Y, X)$, and this affects the output of $f$.

  \begin{proposition}
    Disjointness admits a fooling set of size $2^k$.
  \end{proposition} 
  \begin{proof} 
    Let $F = \{ (X, \bar{X})\ |\ X \subseteq [k] \}$.
    Then $f(A, B) = 1$ for all $(A, B) \in F$. But now consider any $(X,
    \bar{X}), (Y, \bar{Y}) \in F$. It cannot be the case that $X \cap \bar{Y} =
    \emptyset$ \textbf{and} $Y \cap \bar{X} = \emptyset$ unless $Y = X$. The
    former implies that $X \subseteq Y$ and the latter implies that $Y \subseteq
    X$. Therefore, either $f(X, \bar{Y}) = 0$ or $f(Y, \bar{X} ) = 0$.
  \end{proof}

  So both protocols necessarily require communication $k$ to solve
  deterministically. It turns out that both problems are also hard to solve
  \emph{nondeterministically}. Disjointness is also hard to solve with a
  randomized protocol, but Equality can be solved with high probability with very
  low randomized communication. 

  \subsection{Multi-party communication} 
    We'll also talk about multi-party
    communication instead of just two parties - this will be relevant when figuring
    out how good combinatorial auctions we should possibly hope for. Here, there's
    again a boolean function $f$ but it takes as input $X_1, \ldots, X_n$. Various
    input models are studied, but we'll only talk about the \emph{number-in-hand}
    model, where player $i$ knows only $X_i$ and nothing else. 

    We'll also want a canonical hard problem to start with, multi-disjointness. It
    also introduces a new concept, called a \emph{promise problem}.

    \begin{Exa}[Multi-Disjointness] 
      Let $f(X_1,\ldots, X_n) = 1$ iff $X_i \cap X_j = \emptyset$
      for all $i \neq j$. Moreover, we'll be interested in the
      \emph{promise problem}, where the protocol is promised that one of the
      following is true about the input: 
      \begin{itemize} 
        \item $X_i \cap X_j =
        \emptyset$ for all $i \neq j$.  \item $\cap_i X_i \neq \emptyset$ (there
        exists an $\ell$ in all $X_i$).  
      \end{itemize}
      More clearly, we say that a
      communication protocol solves Multi-Disjointness if it outputs $1$ whenever $X_i
      \cap X_j = \emptyset$ for all $i \neq j$,
      outputs $0$ whenever $\cap_i X_i \neq \emptyset$,
      and is allowed to behave arbitrarily if neither condition holds.
    \end{Exa}

    \begin{theorem}[\cite{NisanPackCover02}]
      Every deterministic protocol for
      multi-disjointness has communication complexity $\Omega(k/n)$.
    \end{theorem}
    \begin{proof}
      First, observe that an analog of the rectangles argument holds in
      multi-party communication as well: if $T(X_1,\ldots, X_n) = T(Y_1,\ldots,
      Y_n)$, then $T(Z_1,\ldots, Z_n) = T(X_1,\ldots, X_n)$ whenever $Z_i \in \{X_i,
      Y_i\}$. The proof is the same: if not, there must be some player who is the
      first to send a different message. But their input matches their input in one
      of the two cases $(X_1,\ldots, X_n), (Y_1,\ldots, Y_n)$, so this is a
      contradiction.
      The fooling set argument also extends, but we'll use the same concepts slightly
      differently below.
      % Namely, we count the number of transcripts needed

      First, we count the total number $1$ instances
      (i.e. cases where the input sets are disjoint).
      Observe that each such instance can be selected by choosing, for
      each item, a player to receive that item, or no player at all.
      There are $n+1$ choices, independently for all items,
      giving $(n+1)^k$ possible instances.

      Second, we'll first show that no protocol can use the same
      transcript for more than $n^k$ inputs which evaluate to $1$.
      Let $S$ be a set of inputs, say 
      $\{(X^{(j)}_1,\ldots,X^{(j)}_n)\}_{j=1}^{|S|}$ that
      share the same transcript and are all $1$ instances.
      \begin{claim}
        For each $\ell \in [k]$, there exists a player $i$
        such that player $i$ never gets $\ell$ in $S$
        (i.e. for every input $(X^{(j)}_1,\ldots,X^{(j)}_n)\in S$,
        we have $\ell \notin X^{(j)}_i$)
      \end{claim}
      \begin{proof}
        The negation of the above statement is exactly the following:
        There exists an $\ell\in[k]$ such that for all players $i$,
        $i$ get $\ell$ in some input from $S$, say 
        $\ell \in X^{j(\ell)}_i$.
        But by the rectangle argument, the input
        $ (X^{j(\ell)}_1, \ldots, X^{j(\ell)}_n) $ has the same transcript
        as all the elements of $S$, and thus should be a $1$ instance.
        However, $\ell \in X^{j(\ell)}_1 \cap \ldots \cap X^{j(\ell)}_n$,
        a contradiction.
      \end{proof}
      With this claim in mind, we bound the total possible number of inputs
      in $S$. Consider all the ways to put an item $\ell\in[k]$
      into an input tuple: it may go to no bidder, or it may go
      to exactly one bidder (because the sets of an input tuple are disjoint).
      But, by the above claim, each item has one bidder which it
      \emph{cannot} go to.
      This gives each item exactly $n$ choices,
      for a total size of $S$ of at most $n^k$.

      Finally, observe that the above two results mean that there are at least
      $(n+1)^k/n^k = (1 + 1/n)^k$ different transcripts are needed simply to handle the
      $1$-instances of the problem.
      This gives communication complexity at least
      $k\log_2(1+1/n) = \Theta(k/n)$.
    \end{proof}

\section{Monotone Valuations} 
  First, we'll present a lower bound for monotone
  valuations, by reduction from MultiDisjointness.

  \begin{theorem}[\cite{NisanPackCover02}]
    The communication required to guarantee strictly
    better than an $n$-approximation for welfare maximization with $m$ items on all
    monotone valuations is $2^{\Omega(m/n^2)}$. Note that this implies that
    exponential communication among the bidders is necessary in order to guarantee
    that an allocation is within a factor of $n$ of optimal when $n =
    m^{1/2-\varepsilon}$.
  \end{theorem}
  \begin{proof}
    Let $k$ be an integer we'll tune later.
    First, assume that we have
    a $k$-size family of length-$n$ partitions of items $M$, say
    $\{(T_1^\ell,\ldots, T_n^\ell)\}_{\ell \in [k]}$,
    where for all $\ell$, $T_i^\ell \cap T_j^\ell = \emptyset$.
    Additionally, we require that for
    all $\ell \neq \ell'$, $T_j^\ell \cap T_i^{\ell'} \neq \emptyset$;
    in other words the disjointness property does not mix and match
    across different partitions.
    We will assign bidder $i$ to the (some sub collection of) the sets
    $\{T^\ell_i\}$.
    % The condition that $(T_1^\ell,\ldots, T_n^\ell)$ are disjoint will
    % correspond to situations where every bidder can be satisfied,
    % while the condition that 
    % That is, we
    % can assign each bidder $T_i^\ell$ for the same $\ell$, but not mix and match
    % across different $\ell$s. 

    Now, consider the following class of valuations, called \emph{multi-minded}, or
    \emph{coverage} valuations: bidder $i$ has some subset $X_i \subseteq [k]$, and
    $v_i(S) = 1$ if and only if $T_i^\ell \subseteq S$ for some $\ell \in X_i$ (and
    $v_i(S) = 0$ otherwise). That is, bidder $i$ ``likes'' all the sets $T_i^\ell$,
    for $\ell \in X_i$, and gets value $1$ as long as they get some set they like. 

    Observe that if there exists a single $\ell \in \cap_i X_i$, then it is possible
    to achieve welfare $n$: simply give each bidder set $T_i^\ell$ for the $\ell \in
    \cap_i X_i$. Also, observe that if it is possible for both bidder $i$ and bidder
    $j$ to get value $1$, then the sets that they receive and are interested in must
    be disjoint. For desired sets $T_i^\ell$ and $T_j^{\ell'}$, this can occur
    only if $\ell = \ell'$.
    So if for all $i, j$, $X_i \cap X_j = \emptyset$,
    then there is no way for both bidder $i$ and bidder $j$
    to get non-zero welfare, and the maximum possible welfare is $1$. So by a direct
    reduction to MultiDisjointness, the communication complexity of determining
    whether the welfare is $\geq n$ or $\leq 1$ is at least $\Omega(k/n)$. 

    Now, we just need to figure out how big of a $k$ we can have while still
    guaranteeing a ``collection of partitions''
    construction of the desired form. Consider $k$ random
    partitions of $[m]$. That is, each item is independently and uniformly awarded
    to a bidder, and $T_i^\ell$ is the set of items awarded to bidder $i$ in
    partition $\ell$. Clearly, this satisfies the first property: for all $\ell$,
    the $T_i^\ell$s indeed form a partition, by definition. We just need to make
    sure we don't take too many random partitions so that we have bad luck and some
    $T_i^\ell \cap T_j^{\ell'}= \emptyset$. 

    So consider a fixed $T_i^\ell, T_j^{\ell'}$. What is the probability that they
    don't intersect? This is the probability that for each item $x$, it is either
    not put into $T_i^\ell$ or not put into $T_j^{\ell'}$. This happens with
    probability $1-1/n^2$. So the probability that this happens for every item is
    exactly $(1-1/n^2)^m \approx e^{-m/n^2}$. Now just take a union bound over all
    $(kn)^2$ sets, and we get that the probability that any two sets happen to be
    disjoint by bad luck is $\approx k^2n^2 e^{-m/n^2}$. Therefore, we can take $k =
    2^{\Omega(m/n^2)}$ and get a non-zero probability of success (even if we can't
    find this instance efficiently, it exists, so we can use that for the
    construction and proof). 

    So to wrap up: we need a construction of the form given
    at the beginning of the proof
    for large $k$. The argument above provides a construction for $k =
    2^{\Omega(m/n^2)}$, which completes the proof.
  \end{proof}

  Recall that we have protocols matching this: a $\sqrt{m}$-approximation due
  to~\cite{LaviS05, DobzinskiNS05}, or the trivial $n$-approximation which just
  allocates randomly (both of the previous results would also guarantee an
  $n$-approximation). 

\section{Subadditive Valuations} 

  \begin{theorem}[\cite{DobzinskiNS05}]The
    communication required to guarantee strictly better than a
    $(1/2+1/(2n))$-approximation for welfare maximization with $m$ items on all
    subadditive valuations is $2^{\Omega(m/n^2)}$.  
  \end{theorem} 
  \begin{proof}
    Again, assume that we have the following construction of partitions:
    $\{T_1^\ell,\ldots, T_n^\ell\}_{\ell \in [k]}$, where for all $\ell$,
    $T_i^\ell \cap T_j^\ell = \emptyset$, but for all $\ell \neq \ell'$,
    $T_j^\ell \cap T_i^{\ell'} \neq \emptyset$. That is, we can assign each
    bidder $T_i^\ell$ for the same $\ell$, but not mix and match across
    different $\ell$s.

    Observe that the previous construction is not subadditive. In particular,
    player's have value $1$ for some sets, while having value $0$ for every proper
    subset. The only change we'll make is to bump up the players' values for every
    non-empty subset by $1$. 

    That is, bidder $i$ has some subset $X_i \subseteq [k]$, and $v_i(S) = 2$ if and
    only if $T_i^\ell \subseteq S$ for some $\ell \in X_i$ (and $v_i(S) = 1$
    otherwise). This function is now clearly subadditive: any two non-trivial sets
    have value one, and no set has value more than two. 

    The only difference is in the welfare guarantee we can claim. Now, if we're in
    the disjoint case, we can certainly get welfare $2n$. But even in the
    non-disjoint case, we can get welfare $n+1$ (one player can get a set they like,
    and everyone else gets a single item). So our lower bound is now only $2n/(n+1)$
    instead of $n$.
  \end{proof}

  Recall that we have a 2-approximation (that we didn't see) due
  to~\cite{Feige06}. This is asymptotically tight as $n\rightarrow \infty$, but
  not for small $n$ (actually, the algorithm is tight for small $n$, and the lower
  bound can be improved, see later section). 

\section{XOS Valuations}

  \begin{theorem}[\cite{DobzinskiNS05}]
    The communication
    required to guarantee strictly better than a $1-(1-1/n)^n$-approximation for
    welfare maximization with $m$ items on all XOS valuations is
    $2^{\Omega(m/n^2)}$.  
  \end{theorem}

  \begin{proof} This time, we'll need a slightly stronger construction (actually,
    the same construction works, we just need to appeal to a stronger property).
    This time, assume that we have the following construction of partitions:
    $\{T_1^\ell,\ldots, T_n^\ell\}_{\ell \in [k]}$, where for all $\ell$,
    $T_i^\ell \cap T_j^\ell = \emptyset$, but for all $\ell_1,\ldots, \ell_n$ (all
    distinct) we have $|\cup_i T_i^{\ell_i}| \leq m\cdot (1-(1-1/n)^n)$. 

  Now, consider the following class of valuations. First, for all $\ell$ define an
  additive valuation $v_i^\ell(S) := |S \cap T_i^\ell|$ (that is, value one per
  item in $T_i^\ell$). Next, give each bidder some set $X_i$ of indices, and
  define $v_i(S) := \max_{\ell \in X_i} v_i^\ell(S)$. This is clearly XOS (even
  Binary XOS). 

  Observe again that if there exists a single $\ell \in \cap_i X_i$, then it is
  possible to achieve welfare $m$: simply give each bidder set $T_i^\ell$ for the
  $\ell \in \cap_i X_i$. But now the welfare might be better in the ``no'' case,
  because the bidders will still get value $1$ per item even if they barely
  overlap with their sets. Indeed, the welfare can be as large as $m \cdot
  (1-(1-1/n)^n)$.

  Again, we just need to figure out how big of a $k$ we can have while still
  guaranteeing a construction of the desired form. Again consider $k$ random
  partitions of $[m]$. That is, each item is independently and uniformly awarded
  to a bidder, and $T_i^\ell$ is the set of items awarded to bidder $i$ in
  partition $\ell$. Clearly, this satisfies the first property: for all $\ell$,
  the $T_i^\ell$s indeed form a partition, by definition. We just need to make
  sure we don't take too many random partitions so that we have bad luck and some
  $T_i^\ell \cap T_j^{\ell'}= \emptyset$. 

  Consider a fixed set of distinct indices $\ell_1,\ldots, \ell_n$. Then the
  expected number of items in their union is exactly $1-(1-1/n)^n \cdot m$
  (because for any item to miss the union, it needs to be missed independently $n$
  times). The probability that it deviates from its expectation by more than
  $\varepsilon \cdot m$ is upper bounded by $e^{-\varepsilon^2 m/2}$. So taking a
  union bound over all $\leq k^n$ distinct indices, we get that this holds for all
  indices with probability $e^{-\varepsilon^2 m/2 + n\log k}$. Setting $n \log k =
  \varepsilon^2 m/8$ results in $k \leq e^{O(m\varepsilon^2/n)}$. We could, for
  instance, let $\varepsilon = 1/\sqrt{n}$, giving the bound in the theorem
  statement (but there are other interesting choices of $\varepsilon$ too).
  \end{proof}

\section{XOS Simultaneous Lower Bound} 
  Consider the following construction:

  \begin{enumerate} \item $S$ is drawn uniformly at random from sets of size
      $m/2$.  \item $T$ is drawn uniformly at random from sets of size $m/2$,
      \textbf{conditioned on $|S \cap T| = m/3$.} \item Let $a_1,\ldots, a_k$ be
        uniformly random sets of size $m/2$, conditioned on $|S \cap a_i| = m/3$.
      \item Alice's initial BXOS function has $A(X) = \max_i \{|a_i \cap X|\}$.
      \item Let $b_1,\ldots, b_k$ be uniformly random sets of size $m/2$,
        conditioned on $T \cap a_i| = m/3$.  \item Bob's initial BXOS function has
        $B(X) = \max_i \{|b_i \cap X\}$.  \item Let $Y$ be a uniformly random set
          satisfying $|Y \cap S| = |\bar{Y} \cap T| = m/3$. Let $Z$ be a uniformly
          random set satisfying $|Y \cap S| = |Y \cap T| = m/3$. With probability
          $1/2$, set $v^A(X) = \max\{A(X),|Y \cap X|\}$ and $v^B(X) = \max\{B(X),
          |\bar{Y} \cap X|\}$. With probability $1/2$ set $v^A(X) = \max\{A(X), |Z
          \cap X|\}$ and $v^B(X) = \max\{B(X), |Z\cap X|\}$.  \end{enumerate}

  \begin{theorem}[\cite{BravermanMW18}] No simultaneous protocol with
    subexponential in $m$ communication can beat a $(3/4-1/108)$ approximation on
    the above construction with probability $> 2/3$ for decision.

  \end{theorem}

  We'll present the following steps, without proof.  \begin{enumerate} \item For
        sufficiently small $k$, but $k = exp(m)$, the maximum welfare between $A$
        and $B$ $(3/4-1/108)m$. This is basically because the union of any two
        sets constructed as in steps 1 thru 6 is $(3/4-1/108)m$.  \item If we do
        the first case in step one, the maximum achievable welfare between $v^A$
      and $v^B$ is $m$.  \item In the second case, the maximum achievable welfare
        between $v^A$ and $v^B$ is $(3/4-1/108)m$. This is essentially because the
        additional sets are no better than another random set.  \item In either
        case, it is impossible for Alice to distinguish which of her sets is this
      special one. Ditto for Bob.  \item Therefore, in order to possibly solve the
        decision problem, Alice or Bob would have to convey some information about
        every single one of their sets, which is exponential. (This can be made
        into a formal statement, but we won't prove that in this class).
    \end{enumerate}

  Observe that with two rounds, Alice could first send Bob the set $S$ (which is
  painfully obvious by her construction), and Bob could then determine which set
  he has that is ``special.''

  Also observe that for search, Alice will just pick a random $i$ and keep the
  items in $a_i$ and give the rest to Bob. This will get the right welfare
  approximation in both cases (it will just get different absolute welfare in
  both). 

\section{Subadditive Lower Bound for Two Players}
  Consider the following
  construction, which is well-defined for any collection $S$ of subsets: for a
  given set $T$, say that $T$ is covered by collection $Z$ of sets if $\cup_{z \in
  Z} z \supseteq T$. Let $f_S(T)$ be the minimum number of sets in $S$ that cover
  $T$. 

  \begin{lemma}$f_S(\cdot)$ is subadditive for all $S$.  \end{lemma} \begin{proof}
    Let $Z$ cover $T$ and $Y$ cover $U$. Then clearly $Z \cup Y$ covers $T \cup
    U$. Therefore, if $Z$ and $Y$ are the minimum covers of $T$ and $U$
    respectively, we have that $f_S(T \cup U) \leq |Z|+|Y| = f_S(T) + f_S(U)$.
  \end{proof}

  Now, we wish to further modify $f_S(\cdot)$ into $f_S^\ell(\cdot)$ as follows:
  \begin{itemize} \item If $f_S(X) < \ell/2$, $f_S^\ell(X) = f_S(X)$.  \item If
      $f_S(\bar{X}) < \ell/2$, $f_S^\ell(X) = \ell - f_S(\bar{X})$.  \item If
        neither, then $f_S^\ell(X) = \ell/2$.  \end{itemize}

  Observe that $f_S^\ell(\cdot)$ might not be well-defined (the first two bullets
  might conflict). But for $S$ with the right properties, it is always
  well-defined. 

  \begin{definition}[$\ell$-sparse] We say that $\mathcal{S}$ is
  \emph{$\ell$-sparse} if for all $T_1,\ldots, T_{\ell-1} \in \mathcal{S}$,
  $\cup_j T_j \neq M$. \end{definition}

  \begin{lemma} If $\mathcal{S}$ is $\ell$-sparse, then
  $f^\ell_\mathcal{S}(\cdot)$ is well-defined.  \end{lemma}

  \begin{proof} It is clear that $f^\ell_\mathcal{S}(X)$ is always defined at
    least once. The only way in which $f^\ell_\mathcal{S}(X)$ could be defined
    multiple times is if $\sigma_\mathcal{S}(X) < \frac{\ell}{2}$ (in which case
    $f^\ell_\mathcal{S}(X) = \sigma_\mathcal{S}(X)$) and
    $\sigma_\mathcal{S}(\overline{X}) < \frac{\ell}{2}$ (in which case
    $f^\ell_\mathcal{S}(X) = \ell - \sigma_\mathcal{S}(\overline{X})$). 

  So assume for contradiction that both events hold, and let $X \subseteq
  \cup_{i=1}^{\ell/2-1} T_i$, and $\overline{X} \subseteq \cup_{i=1}^{\ell/2-1}
  U_i$, where each $T_i, U_i \in \mathcal{S}$. Note that $\sigma_\mathcal{S}(X),
  \sigma_\mathcal{S}(\overline{X}) < \ell/2$ implies that such $T_i, U_i$ must
  exist. But now consider that we can write $M = X \cup \overline{X}$ as a union
  of $\leq \ell -2$ elements of $\mathcal{S}$, contradicting that $\mathcal{S}$ is
  $\ell$-sparse.  \end{proof} That is, $\mathcal{S}$ is $\ell$-sparse if there
  does not exist $\ell-1$ elements of $\mathcal{S}$ whose union is the entire
  ground set $M$. 

  \begin{proposition} If $\mathcal{S}$ is $\ell$-sparse, then
  $f^\ell_\mathcal{S}(\cdot)$ is monotone and subadditive.  \end{proposition}

  (Proof omitted). So now we have a class of subadditive functions. We now wish to
  come up with a hard instance. To this effect, we need one further definition.

  \begin{definition}[\cite{KleitmanS73}] A collection $\mathcal{S} = \{S_1,\ldots,
  S_k\}$ is \emph{$\ell$-independent} if $\{T_1,\ldots, T_k\}$ is $\ell$-sparse
  whenever $T_i \in \{S_i, \overline{S_i}\}$.  \end{definition}

  \begin{proposition}\label{prop:lindependent}Let $\mathcal{S}$ be an
  $\ell$-independent collection with $|\mathcal{S}| = k$. Then any deterministic
  communication protocol that guarantees a $(\frac{1}{2} +
  \frac{1}{2\ell-3})$-approximation to the optimal welfare for two monotone
  subadditive bidders requires communication at least $k$.  \end{proposition}

  \begin{proof} Let $\mathcal{S} = \{S_1,\ldots, S_k\}$ be $\ell$-independent. For
    each $i$, define $S_i^1:= S_i$, and $S_i^0 := \overline{S_i}$. Now, consider
    an instance of EQUALITY where Alice is given $a$ and Bob is given $b$. Alice
    will create the valuation function $f^\ell_\mathcal{A}$, where
    $\mathcal{A}:=\{S_1^{a_1},\ldots, S_k^{a_k}\}$ (i.e. Alice builds
    $\mathcal{A}$ by taking either $S_i^1$ or $S_i^0$, depending on $a_i$). Bob
    will create the valuation function $f^\ell_\mathcal{B}$, where
    $\mathcal{B}:=\{S_1^{b_1},\ldots, S_k^{b_k}\}$. Observe first that
    $f^\ell_\mathcal{A}(\cdot)$ and $f^\ell_\mathcal{B}(\cdot)$ are indeed
    well-defined, monotone, and subadditive as $\mathcal{S}$ is $\ell$-independent
    (and therefore $\mathcal{A}$ and $\mathcal{B}$ are both $\ell$-sparse). 

  Observe that if $a = b$, then $\mathcal{A} = \mathcal{B}$ and moreover
  $f^\ell_\mathcal{A}(\cdot) = f^\ell_\mathcal{B}(\cdot)$. Observe that for any
  $S, \ell$ we have $f_S^\ell(X) + f_S^\ell(\bar{X}) =\ell$, so the maximum
  achievable welfare is $\ell$.

  On the other hand, if there exists an $i$ such that $a_i \neq b_i$ (without loss
  of generality say that $a_i = 1$ and $b_i = 0$), we claim that welfare $2\ell-2$
  is achievable. To see this, consider the allocation which awards
  $\overline{S_i}$ to Alice and $S_i$ to Bob. Indeed, $f^\ell_\mathcal{A}(S_i) =
  1$ (as $S_i \in \mathcal{A}$), so $f^\ell_\mathcal{A}(\overline{S_i}) = \ell -
  1$. Similarly, $f_B(\overline{S_i}) = 1$, so $f_B(S_i) = \ell - 1$, achieving
  total welfare $2(\ell-1)$.\footnote{
    As an aside, note that welfare exceeding
    $2\ell-2$ is not possible, as Alice and Bob each value all non-empty sets at
    least at $1$, and therefore value all strict subsets of $M$ at most at $\ell-1$
    by \ref{lminus}. Therefore, the only way Alice or Bob could have value exceeding
    $\ell-1$ is to get all of $M$, meaning that hte other player receives value
    $0$. } 

  So assume for contradiction that a deterministic $\frac{1}{2} +
  \frac{1}{2\ell-3} > \frac{\ell}{2\ell-2}$-approximation exists to the optimal
  welfare for two monotone subadditive bidders with communication $< k$. Then such
  a protocol would solve EQUALITY with communication $<k$ by the reduction above,
  a contradiction.  \end{proof}

  \begin{lemma} \label{lem:l-indep-exist} For all $m$, $x > 1$, and $\ell =
  \log_2(m) - \log_2(x)$, there exists a $\ell$-independent collection of subsets
  of $[m]$ of size $k = e^{x/\ell}$.  \end{lemma}

\section{Possible Project Topics} 
  \begin{itemize} \item See if the presented
      subadditive construction ``separates'' XOS from subadditive in other cases
      where good algorithms for subadditive aren't known but good ones for XOS
      are (ask me for examples).
  \end{itemize}

\bibliographystyle{alpha} \bibliography{../MasterBib}


\end{document}
